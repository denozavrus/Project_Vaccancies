\documentclass[12pt]{article}
\usepackage{fontspec}
\usepackage[utf8]{inputenc}
\usepackage{titlesec}
\usepackage[russian]{babel}
\usepackage{indentfirst}
\usepackage{longtable}
\usepackage{setspace}
\usepackage{hyperref}
\setmainfont{Calibri}
\usepackage[a4paper, margin=1in]{geometry}
\usepackage{blindtext}
\usepackage{draftwatermark}
\begin{document}
\pagestyle{empty}
\SetWatermarkText{В РАЗРАБОТКЕ}
\SetWatermarkScale{0.8}
\spacing{1.15}
\begin{center} \textbf{Университет науки и технологий МИСИС}

  Направление <<09.04.01 ИНФОРМАТИКА И ВЫЧИСЛИТЕЛЬНАЯ ТЕХНИКА>>

  Профиль «Интеллектуальные программные решения для бизнеса»

  \vspace{1.0cm}
  Отчет о самостоятельной работе по \\ дисциплине «Программная инженерия (Python)»

\end{center}

\vspace{10.0cm}
\begin{flushright}
{Бригада № 2}:

Лазаренко Д.~М., 1 курс, группа МИВТ-22-5

Маковецкий И.~А., 1 курс, группа МИВТ-22-5

\end{flushright}

\vspace*{\fill}

\begin{center}
Москва 2022
\end{center}

\newpage
\pagestyle{plain}
\renewcommand{\contentsname}{Оглавление}
\tableofcontents
\newpage

\section{Общая постановка задачи}

\subsection{Описание прикладной области и данных}
Выбранная прикладная область ---  «Уровень предлагаемых зарплат технических специалистов в России». Задачей данного исследования является анализ и прогнозирование уровня заработной платы сотрудников по ряду признаков.

Информация, используемая в данном исследовании была собрана из открытых источников (см. \ref{sec:app1}).

В таблице~\ref{tab:factors-desc} представлено описание фактов, учтенных в анализе.
\begin{longtable}{|l|l|l|l|l|}
\caption{Описание фактов, учтенных в анализе}
\label{tab:factors-desc}\\
\hline
№ & \begin{tabular}[c]{@{}l@{}}Характеристика объекта/ \\ явления\end{tabular} & \begin{tabular}[c]{@{}l@{}}Название\\ переменной\end{tabular} & \begin{tabular}[c]{@{}l@{}}Шкала\\ объяснения\end{tabular} & \begin{tabular}[c]{@{}l@{}}Роль:\\ целевая/\\ объясняющая\end{tabular} \\ \hline
\endfirsthead
%
\multicolumn{5}{c}%
{{Tаблица \thetable\ --- продолжение}} \\
\hline
№ & \begin{tabular}[c]{@{}l@{}}Характеристика объекта/ \\ явления\end{tabular} & \begin{tabular}[c]{@{}l@{}}Название\\ переменной\end{tabular} & \begin{tabular}[c]{@{}l@{}}Шкала\\ объяснения\end{tabular} & \begin{tabular}[c]{@{}l@{}}Роль:\\ целевая/\\ объясняющая\end{tabular} \\ \hline
\endhead
%
1 & Заработная плата & salary\_from, salary\_to & Порядковая & Целевая \\ \hline
2 & Адрес места работы  & coordinates  & Номинальная & Объясняющая \\ \hline
3 & Сопроводительное письмо & response\_letter  & Отношений & Объясняющая \\ \hline
4 & Город & city & Номинальная & Объясняющая\\ \hline
5 & Широта & longitude &  Номинальная & Объясняющая\\ \hline
6 & Долгота & latitude  & Номинальная & Объясняющая\\ \hline
7 & Необработанный адрес & raw & Номинальная & Объясняющая\\ \hline
8 & Опыт & experience & Порядковая & Объясняющая\\ \hline
9 & Время работы & schedule, employment & Номинальная & Объясняющая\\ \hline
10 & Ключевые навыки & skills & Номинальная & Объясняющая\\ \hline
11 & Проверенный работодатель& has\_test & Отношений & Объясняющая\\ \hline
12 & Заработная плата до вычета& gross & Порядковая & Объясняющая\\ \hline
13 & Валюта & currency & Номинальная & Объясняющая\\ \hline
14 & Премиум-аккаунт & premium & Номинальная & Объясняющая\\ \hline

\end{longtable}

В анализе присутствуют 9 номинальных переменных, 3 порядковых и 1 отношений. Зависимая переменная --- «Заработная плата».
\subsection{Основные гипотезы, которые планируется проверить в рамках исследования}

Для дальнейшего анализа были сформулированы 3 гипотезы о статистической взаимосвязи целевой переменной и объясняющих:
\begin{enumerate}
\item Средняя зарплата на вакансиях с требованием знания английского выше, чем без такого требования. 
\item  При росте опыта работы зарплата разработчиков с навыками JavaScript растет быстрее чем для разработчиков с навыками 1С.
\item  При росте опыта зарплата механиков растет быстрее, чем токарей.
\end{enumerate}
\section{Предварительный анализ собранных данных}
\subsection{Анализ особенностей данных: потенциальные ошибки и пропущенные значения,
группы и выбросы}

\subsubsection{Анализ количественных переменных}
Здесь необходимо построить и проанализировать гистограммы для всех
количественных (интервальных и относительных) переменных в анализе. Необходимо
охарактеризовать вид распределения по отношению к нормальному распределению —
асимметрию, эксцесс, полимодальность. Для этого следует привести график гистограммы
совместно с графиком плотности нормального распределения, а также таблицу основных
статистик.
% Please add the following required packages to your document preamble:
% \usepackage{longtable}
% Note: It may be necessary to compile the document several times to get a multi-page table to line up properly
\begin{longtable}{|l|l|}
\caption{Описание фактов, учтенных в анализе}
\label{tab:statistical-features}\\
\hline
Статистика & Значение \\ \hline
\endfirsthead
%
\multicolumn{2}{c}%
{{Tаблица \thetable\ --- продолжение}} \\
\hline
Статистика & Значение \\ \hline
\endhead
%
Среднее &  \\ \hline
Медиана &  \\ \hline
Стандартное отклонение &  \\ \hline
Межквартильный размах&  \\ \hline

Верхняя квартиль&  \\ \hline

Нижняя квартиль&  \\ \hline

Коэффициент асимметрии&  \\ \hline

Коэффициент эксцесса&  \\ \hline

Количество наблюдений&  \\ \hline

Количество пропущенных значений&  \\ \hline

\end{longtable}

Необходимо дать интерпретацию статистических свойств количественных переменных в контексте предметной области. Например, на основании гистограммы и числовых характеристик распределения можно сделать вывод о наличии небольшого количества субъектов федерации с очень большой долей бедного населения. Также, для целевой переменной следует проанализировать наличие выбросов на основании правила «трех-сигм». Следует отметить в базе все выбросы и на основании сравнения соответствующих значений объясняющих переменных с их средними или медианными значениями объяснить, почему эти наблюдения могут
интерпретироваться как выбросы.

\subsubsection{Анализ качественных переменных}
Здесь следует привести столбчатые диаграммы, которые отражают количество измерений с разными уровнями для данной переменной.

Необходимо проанализировать степень представленности всех уровней и при
необходимости (наличии уровней с долей менее 5 \%) произвести укрупнение уровней.

Результат привести на новых диаграммах. Принцип укрупнения пояснить.

\subsection{Анализ статистической связи}
\subsubsection{Графический анализ пары «целевая переменная  --- качественная объясняющая переменная»}
Здесь для каждой пары (количественная зависимая переменная --- качественная
независимая переменная) необходимо построить категорированную диаграмму Бокса-
Уискера (Box-Whisker).

На основании анализа диаграммы следует охарактеризовать связь среднего
значения и разброса количественной зависимой переменной с уровнями качественной
независимой переменной. Интерпретацию дать в контексте предметной области.

Для формальной проверки гипотезы о наличии статистической связи следует
выполнить непараметрический дисперсионный анализ (критерий Крускала-Уоллиса)

\subsubsection{Графический анализ пары «числовая зависимая переменная – числовая
независимая переменная»}
Здесь для каждой пары (количественная зависимая переменная – количественная
независимая переменная) необходимо построить диаграммы рассеивания (Scatter plot).

На основании визуального анализа диаграммы следует сделать предположение о
наличии и характере статистической взаимосвязи. Интерпретацию результатов дать в
контексте предметной области.

Для формальной проверки гипотезы о наличии связи следует подсчитать
коэффициенты корреляции Пирсона и Спирмена, а также тау Кендала и привести
результаты проверки их значимости.

\subsubsection{Анализ статистической взаимосвязи между независимыми переменными}

Следует проанализировать силу связи между независимыми переменными,
используя инструменты пп. 3.2.1 и 3.2.2. Для анализа силы связи между качественными
переменными следует использовать анализ таблиц: необходимо привести таблицу кросс-
табуляции, значения статистики хи-квадрат и V-Крамера.

\subsubsection{Предварительная проверка гипотез}
Здесь необходимо рассказать о результатах проверки гипотез из п.1.3 на основании
предварительного анализа данных.

\section{Проверка гипотез с помощью моделирования}
Данный раздел предполагает проверку прогностических способностей построенной
модели. В связи с этим исходную выборку следует случайным образом разделить на
обучающую и тестовую в пропорции 80:20. На обучающей выборке будет осуществляться
построение моделей, тестовая выборка будет использоваться для проверки
прогностических способностей.

\subsection{Построение базовой модели}
Базовая модель служит для анализа изменения качества моделирования при учете
сформулированных гипотез. В качестве базовой модели следует использовать модель
линейной регрессии целевой переменной на все объясняющие. Для базовой модели
следует проверить значимость всех объясняющих переменных, а также уровень
мультиколлинеарности (показатель VIF) и наличие гетероскедастичности (критерий Уайта).
Исходная базовая модель и результаты ее анализа включается в отчет.

Далее необходимо оптимизировать структуру модели для повышения ее качества и
возможного снижения уровня мультиколлинеарности. Для этого следует пошагово удалять
незначимые переменные, переоценивая модель после каждого удаления. Необходимо
также пошагово удалять переменные, которые демонстрируют высокую взаимосвязь с
другими переменными (VIF > 3). Оценку мультиколлинеарности и гетероскедастичности
следует выполнять на каждом шаге оптимизации. В отчете следует привести один
промежуточный и итоговый вариант, который не содержит незначимых объясняющих
переменных и имеет удовлетворительный уровень мультиколлинеарности. Следует
привести оценку мультиколлинеарности вошедших в модель переменных и оценку
наличия гетероскедастичности. Необходимо также привести оценку качества полученной
модели (критерий Akaike, R-sq и adjusted R-sq).

В ходе оптимизации следует оставить в модели объясняющие переменные, которые
необходимы для проверки гипотез даже, если они незначимы или имеют высокое
значение показателя VIF. Это следует отметить в отчете.

\subsection{Проверка гипотез с помощью моделирования}
Для проверки выдвинутых в п. 1.2. сложных гипотез выполняется модификация
оптимизированной базовой модели поэтапно для каждого сочетания сформулированных
гипотез. Сначала модифицируют базовую модель для каждой сложной гипотезы отдельно,
далее для всевозможных пар и т.д. Для простых гипотез модификация не требуется.
Модифицированные модели оцениваются и выполняется проверка как сложных, так и
простых гипотез. Методология проверки каждой гипотезы должна быть описана в отчете в
виде ограничений на коэффициенты и пары статистических гипотез. Результаты
использования каждой модифицированной модели включаются в отчет.

Модель, которая учитывает все сформулированные гипотезы объявляется итоговой.
\subsection{Оптимизация итоговой модели, сравнение качества моделей}
Итоговая модель подвергается оптимизации за счет пошагового удаления
незначимых переменных. На каждом шаге модель переоценивается. Для финального
варианта оценивается качество модели с использованием критерия Akaike и adjusted R-sq.
Оптимизированная итоговая модель и результаты ее анализа включаются в отчет.

По результатам работы формируется таблица с перечнем моделей включенных в
отчет и оценками их качества — значениями критерия Akaike, R-sq и adjusted R-sq

\begin{longtable}{|l|l|l|l|}
\caption{Сравнение качества построенных моделей}
\label{tab:quality-desc}\\
\hline
Номер или критерий  & \begin{tabular}[c]{@{}l@{}}$R^2$\end{tabular} & \begin{tabular}[c]{@{}l@{}}${Adj} \setminus R^{2}$\end{tabular} & \begin{tabular}[c]{@{}l@{}} Akaike \end{tabular} \\ \hline
\endfirsthead
%
\multicolumn{5}{c}%

{Tаблица \thetable\ --- продолжение} \\

\hline
№ & \begin{tabular}[c]{@{}l@{}}Характеристика\\ объекта/явления\end{tabular} & \begin{tabular}[c]{@{}l@{}}Название\\ переменной\end{tabular} & \begin{tabular}[c]{@{}l@{}}Шкала\\ объяснения\end{tabular} & \begin{tabular}[c]{@{}l@{}}Роль:\\ целевая/\\ объясняющая\end{tabular} \\ \hline
\endhead
%
1   &  &  &  \\ \hline
2   &  &  &  \\ \hline
3   &  &  &  \\ \hline
\end{longtable}

\subsection{Проверка прогностических способностей модели}
Проверка прогностических способностей осуществляется для всех включенных в
отчет моделей. Необходимо подсчитать значения прогнозов для элементов тестовой
выборки и построить для них центральные доверительные интервалы на основе
нормального распределения для доверительной вероятности 95\%. Для результатов
следует рассчитать среднеквадратическую погрешность прогнозирования и максимальную
абсолютную погрешность прогнозирования, а также эмпирическую оценку доверительной
вероятности. Результаты следует представить в виде таблицы

\begin{longtable}{|l|l|l|l|}
\caption{Сравнение прогностических способностей моделей}
\label{tab:vanga-desc}\\
\hline
Номер или критерий  & \begin{tabular}[c]{@{}l@{}}Среднеквадратичная \\ погрешность$\end{tabular} & \begin{tabular}[c]{@{}l@{}} Абсолютная \\ погрешность \end{tabular} & \begin{tabular}[c]{@{}l@{}} Доверительная \\ вероятность \end{tabular} \\ \hline
\endfirsthead
%
\multicolumn{5}{c}%
{Tаблица \thetable\ --- продолжение} \\
\hline
№ & \begin{tabular}[c]{@{}l@{}}Характеристика\\ объекта/явления\end{tabular} & \begin{tabular}[c]{@{}l@{}}Название\\ переменной\end{tabular} & \begin{tabular}[c]{@{}l@{}}Шкала\\ объяснения\end{tabular} & \begin{tabular}[c]{@{}l@{}}Роль:\\ целевая/\\ объясняющая\end{tabular} \\ \hline
\endhead
%
1   &  & 0-10 & Целевая \\ \hline
2   &  &  &  \\ \hline
3   &  &  &  \\ \hline
\end{longtable}
Таблицу следует прокомментировать, в частности, оценку доверительной
вероятности. Результаты, представленные в таблице, следует сопоставить с оценками
качества данных моделей.
\subsection{Диагностика регрессионной модели}
Для оптимизированной базовой модели и для оптимизированной итоговой модели
необходимо выполнить поиск:
\begin{itemize}
  \item точек разбалансировки с помощью hat-value;
        \item выбросов, с помощью стьюдентизированных остаточных разностей;
        \item измерений сильно влияющих на оценки коэффициентов, с помощью
расстояния Кука.
\end{itemize}
Необходимо сравнить полученные множества для двух моделей и выделить
измерения, которые входят в указанные множества для обоих моделей.

Также, необходимо проанализировать несколько точек (две — три) с аномальными
значениями расстояния Кука для оптимизированной итоговой модели. Следует установить,
входят ли они в множества точек разбалансировки и выбросов, а также проанализировать,
чем это объясняется. Для этого следует сравнить значения объясняющих и целевой
переменных сор средними значениями по всей выборке.

\section{Заключение}
В данном разделе следует перечислить результаты проверки сформулированных
гипотез в различных сочетаниях, проверки прогностических способностей моделей и их
диагностики.

\appendix
\titleformat{\section}[display]
  {\normalfont\Large\bfseries}
  {\centering \thesection}
  {0pt}{\Large\centering}
\renewcommand{\thesection}{Приложение \Asbuk{section}}

\newpage
\section{}
\label{sec:app1}

\end{document}

% Local Variables:
% TeX-engine: xetex
% End:
